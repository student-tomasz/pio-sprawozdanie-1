\documentclass[10pt,a4paper]{article}
\usepackage[a4paper]{geometry}

\usepackage{polski}
\usepackage{xltxtra}

\usepackage{fancyvrb}
\usepackage{relsize}
\usepackage[pdfborder={0 0 0}]{hyperref}
\usepackage{booktabs}
\usepackage{eurosym}

%% tweak fonts
\defaultfontfeatures{Mapping=tex-text}
\setromanfont{Charis SIL}
%\setsansfont[Scale=MatchLowercase]{Gill Sans}
%\setmonofont[Scale=MatchLowercase]{Menlo}
\linespread{1.25}

%% define custom commands and environments
\DefineVerbatimEnvironment%
  {SmallVerbatim}%
  {Verbatim}{fontsize=\relsize{-0.5},numbers=left,numbersep=-10pt,frame=lines,tabsize=4}

\begin{document}

%%fakesection{Tytuł}
\title{
  Sprawozdanie nr~1 z~laboratorium\\Podstaw Inżynierii Oprogramowania
}
\author{
  Grzegorz Bartkowiak\\
  Tomasz Cudziło\\
  Mateusz Ochtera\\
  Gustaw Wypych\\
  \\
  \textsc{PW EE Informatyka}\\[10pt]
}
\date{\today}

\maketitle

\section{Sytuacja biznesowa~--- Dom aukcyjny \emph{Hammer}}
Nowy zarząd domu aukcyjnego \emph{Hammer} postanowił zinformatyzować swoje
przedsiębiorstwo. Do tej pory całą dokumentacja finansowa, kadrowa, spisy
przedmiotów do sprzedaży jak i sprzedanych były w postaci papierowej. Cały
proces zarządzania dokumentacją był mało wydajny, a~jego utrzymanie wymagało
sporych nakładów finansowych. Zarząd podjął decyzję zastąpienia starego
systemu, nowym~--- wykonanym na zlecenie~--- całkowicie zinformatyzowanym
systemem, który zarchiwizuje całą, dotychczas zebraną, dokumentację
i~jednocześnie przejmie odpowiedzialność za prowadzenie nowej.

Idąc za ciosem, skoro już wszystko będzie dostępne elektronicznie, kierownik
działu sprzedaży przekonał zarząd do wkroczenia na rynek internetowy. Pod
dotychczasową domeną domu aukcyjnego \emph{hammer.com} ma powstać
internetowy system aukcyjny dający możliwość brania udziału w aukcjach w czasie
rzeczywistym.

Ponieważ system zarządzający dokumentacją nie wymaga niestandardowych
rozwiązań, zarząd zlecił jego wykonanie grupie sprawdzonych informatyków z~ich
siostrzanej firmy handlującej krasnalami ogrodowymi. Natomiast internetowy
system aukcyjny potrzebuje nietuzinkowego podejścia.

Dział marketingu wie, że targetem domu aukcyjnego \emph{Hammer} są młodzi
i~przedsiębiorczy ludzie, do których nie przemawiają stare (pomimo, że
sprawdzone) rozwiązania. Zarząd, wspierając się radami z~marketingu,
postanowił, że kluczem do sukcesu jest jak ``najświeższy'' pomysł. Tylko młoda
grupa ludzi, doświadczona w~używaniu i~tworzeniu oprogramowania,
wykorzystującego nowe technologie na różnych platformach mobilnych
i~desktopowych jest w~stanie stworzyć najlepszy system aukcji internetowych.

Platforma aukcyjna ma udostępniać dotychczasowym i~nowym klientom domu
aukcyjnego \emph{Hammer} przeprowadzanie i~uczestniczenie w~aukcjach. Musi być
bezpieczna i~wygodna w~obsłudze.

Poszukiwania zaczęto od czołowych polskich uczelni technicznych. Po sprawdzeniu
rankingów lat ubiegłych zdecydowano się na Politechnikę Warszawską. Ogłoszono
konkurs mający wyłonić grupę informatyków, która zajmie się zaprojektowaniem,
stworzeniem i~wdrożeniem internetowego systemu aukcyjnego \emph{Hammer.com}.

\noindent Postawiono jasne kryteria oceniania pomysłów konkursowych.
Najważniejszymi były:
\begin{itemize}
  \item prostota i~intuicyjność obsługi systemu od strony pracowników oraz
    klientów,
  \item bezpieczeństwo prowadzonych aukcji.
\end{itemize}
Położono też nacisk na:
\begin{itemize}
  \item oryginalność pomysłów,
  \item wydajność i~małą złożoność wewnętrznych mechanizmów serwisu aukcyjnego,
  \item czytelność graficznego interfejsu użytkownika.
\end{itemize}   

Wygranym zespołem okazali się młodzi studenci informatyki, pierwszego roku na
Wydziale Elektrycznym, których wizjonerski i modernistyczny pomysł wprawił
zarząd w zachwyt.

Zwycięzcy informatycy mają za zadanie zaimplementować i~wdrożyć swój projekt,
we współpracy z~zespołem odpowiedzialnym za wewnętrzny system zarządzający
dokumentacją domu aukcyjnego. Współpraca ma polegać na konsultacjach
i~stopniowym wdrażaniu wizji w~życie:
\begin{enumerate}
  \item przedstawienie logotypu domu aukcyjnego, projektu graficznego
    statycznych stron domu aukcyjnego i~szablonów dla dynamicznych stron
    systemu aukcyjnego,
  \item kod udostępniający dane pracowników, klientów, przedmiotów i~aukcji:
  \begin{enumerate}
    \item zaprojektowanie \texttt{API} do komunikacji zewnętrznego systemu
      aukcyjnego z~wewnętrznym systemem z~danymi pracowników, klientów,
      przedmiotów i~aukcji,
    \item \emph{opcjonalnie} jeśli zaprojektowanie \texttt{API} jest
      nieodpowiednim rozwiązaniem, przeorganizowanie wewnętrznego systemu tak
      by baza danych była bezpośrednio dostępna przez serwis aukcyjny i~jej
      tablice odpowiadały wymaganiom systemu wewnętrznego i~systemu aukcji
      internetowych.
  \end{enumerate}
  \item kod sterujący aukcjami i~komunikacją z~klientami,
  \item kod zabezpieczający przed działaniami nielegalnymi i/lub~niezgodnymi
    z~regulaminem domu aukcyjnego,
  \item przeprowadzenie testów.
\end{enumerate}

\section{Przygotowanie do projektu}

\subsection{Skład i~możliwości zespołu}

% \begin{center}
% \begin{tabular}{llr}
%   \toprule
%   Członek zespołu & Stanowisko & Zdolności \\
%   \midrule
%   Adam Ktoś & Grafik/Webdesigner & \\
%   Zbysiu Innyktoś & Programista/Webdeveloper & \\
%   Czesław Niktgoniezna & Administrator baz danych & \\
%   John Tegonatomiastkazdyzna & Prawnik/Konsultant & \\
%   \bottomrule
% \end{tabular}
% \end{center}
\begin{description}
  \item[Artur Ktoś]~---\quad programista front\dywiz end \hfill \\
    Doświadczony w~tworzeniu stron internetowych zgodnych z~standardami
    \texttt{HTML5}, \texttt{CSS3} z~dodatkiem \texttt{AJAX}, głównie za pomocą
    biblioteki \texttt{jQuery}.
  \item[Zbysiu Inny]~---\quad programista back\dywiz end \hfill \\
    Zna \texttt{Javę}, \texttt{Ruby} oraz web\dywiz framework \texttt{Ruby on
    Rails}.  Dobrze zna środowisko open\dywiz source skupione wokół
    \texttt{Ruby}. Zna frameworki do pisania testów: \texttt{RSpec}, oraz
    bardziej przyjazny nie\dywiz programistom \texttt{Cucumber}. Brał udział
    w~kilku projektach, potrafi prowadzić mały zespół.
  \item[Czesław Niktgoniezna]~---\quad grafik, specjalista UX \hfill \\
    Zna \texttt{GIMP} oraz {\tt Adobe Photoshop} i~\texttt{Adobe InDesign}.
    Jest w~stanie przygotować grafikę do druku oraz na ekran. Razem ze Zbyśkiem
    współtworzył inne projekty, jest w~stanie zaprojektować interfejs
    użytkownika od strony graficznej i~użytkowej na urządzenia mobilne i~na
    desktop.
  \item[John Tegonatomiastkazdyzna]~---\quad admin baz danych, admin serwerów, tester \hfill \\
    Zajmuje się administracją lokalnego serwera ze środowiskami do
    programowania, testowania i~zewnętrznym \texttt{VPS} ze środowiskiem
    produkcyjnym. Obsługuje bazy danych \texttt{SQL} oraz słyszał
    o~\texttt{NoSQL}. Potrafi wykorzystać \texttt{Python} i~\texttt{Perl} do
    zarządzania systemami \texttt{Linux}\dywiz podobnymi.
\end{description}

\subsection{Technologie}
Zespół posługuje się głównie wolnym i~otwartym oprogramowaniem, co pozwala na
znaczne obniżenie kosztów początkowych oraz wymusza stosowanie sprawdzonych rozwiązań:
\begin{itemize}
  \item języki programowania i~markup: \texttt{Ruby}, \texttt{Java},
    \texttt{JavaScript}, \texttt{HTML5}, \texttt{CSS3};
  \item frameworki:
    \begin{itemize}
      \item treść: \texttt{Ruby on Rails};
      \item testowanie i~specyfikacja: \texttt{RSpec}, \texttt{JUnit},
        \texttt{Cucumber};
    \end{itemize}
  \item narzędzia: \texttt{GIMP}, \texttt{Vim}, \texttt{Eclipse};
  \item kontrola wersji: \texttt{Git}.
\end{itemize}

\subsection{Wymagania w stosunku do kontrahentów}
\label{sec:wymagania_od_kontrahenta}
Firma \emph{Hammer} udostępni na swój koszt, prawnika który, w roli
konsultanta, pomoże ustalić legalność wprowadzanych opcji w interfejsie
użytkownika. Sformułuje również regulamin na potrzeby serwisu aukcyjnego,
zgodny z prawem obowiązującym na docelowym rynku (mogą być wymagane usługi
tłumaczy).

Firma \emph{Hammer} jest również zobowiązana pokryć koszta związane
z~rejestracją domeny, wynajęciem \texttt{VPS} lub serwerów dedykowanych
i~reszty opłat związanych z~hostingiem. Potrzebne jest również prywatne
repozytorium kodu na \texttt{github.com} by przekazać kod po ewentualnym
zakończeniu współpracy.

\subsection{Czas i~koszt wykonania projektu}
Wstępny projekt podglądowy (interfejs graficzny) zostanie wykonany w przeciągu
dwóch tygodni. Jeżeli zostanie zaakceptowany, przez kolejne sześć tygodni
zostanie napisane kod obsługujący stronę. Przewidziany jest dodatkowy tydzień
opóźnienia na ewentualne poprawki. W~najlepszym przypadku projekt będzie
prowadzony zwinnie przez \emph{test driven development}.

Cały projekt został wyceniony na \EUR{20000}. Projekt prowadzony zwinnie
powinien wykluczyć wszelkie rozbieżności pomiędzy końcowym produktem
a~oczekiwaniami klienta. Mimo to wszelkie poprawki, wynikające z
niezaimplementowania zadeklarowanej funkcjonalności, zespół jest zobowiązany
wykonać na własny koszt. Zmiany interfejsu graficznego po zaakceptowaniu
projektu wstępnego będą dodatkowo płatne po ustaleniach indywidualnych.

Sumarycznie cały projekt powinien być wykonany w~przeciągu 8 do 9 tygodni, za
łączną wartość \EUR{20000} plus koszty opisane
w~par.~\ref{sec:wymagania_od_kontrahenta}. Dalsza współpraca ze zleceniodawcą
zależy od efektów pracy zespołu.

\section{Koordynacja zadań członków zespołu}

Pierwszym zadaniem jest stworzenie szaty graficznej dla nowych stron domu
aukcyjnego. Za prace nad tą częścią odpowiada w~całości \textbf{Czesław
Niktgoniezna}.  W~czasie dwóch tygodni Czesław, w~ścisłej współpracy
z~\textbf{konsultantem z~działu marketingu}, tworzy szablon dla całego serwisu.
Przewidziane w~tym czasie są cztery spotkania z~kierownikiem działu marketingu
w~celu akceptacji dotychczas wykonanej pracy. Czwarte spotkanie ma zakończyć
się podpisaniem dokumentu, w~którym kierownik marketingu akceptuje szatę
graficzną. W~przeciwnym przypadku, możliwe jest przedłużenie procesu nie na
koszt zespołu. Ukończenie nie jest niezbędne do zaczęcia prac nad kodem.

Jednocześnie {\bf John Tegonatomiastkazdyzna} i~{\bf Zbysiu Inny}
w~porozumieniu z~{\bf adminem baz danych z~zespołu zajmującego się systemem
wewnętrznym} aktualizują tablice bazy danych tak by mogła być używana
jednocześnie przez obydwa systemy. Przeznaczone są na to dwa tygodnie, nie
przewidziano opóźnień.

Gdy prace nad bazą danych zostaną zakończone John Tegonatomiastkazdyzna
tworzy środowisko developerskie i~do przeprowadzania testów z~kopiami bazy
danych domu aukcyjnego. {\bf Zbysiu Inny} i~{\bf Artur Ktoś} zaczynają w~tym
momencie prace na kodem zarządzającym aukcjami internetowymi. {\bf Zbysiu}
konsultuje się regularnie z {\bf prawnikiem} (par.
\ref{sec:wymagania_od_kontrahenta}) w~miarę wprowadzanych funkcji. Po jego
akceptacji, przedstawia co tydzień postępy bezpośrednio przed zarządem. Ta
część trwa sześć tygodni.

Przez ostatnie dwa tygodnie {\bf Zbysiu Inny} i~{\bf Artur Ktoś} pracują nad
zabezpieczeniami aukcji prowadzonych przez Internet. {\bf John
Tegonatomiastkazdyzna} konfiguruje serwer produkcyjny i~po zakończeniu prac,
uruchamia na nim system. Nad postępami czuwa konsultant wyznaczony przez
zarząd, prawdopodobnie pracownik zewnętrznej firmy konsultingowej. Jego
akceptacja oznacza zakończenie prac nad systemem aukcji internetowych.

W~przypadku {\bf krytycznych decyzji} wpływających na cały system, problem
przedstawiany jest liderowi zespołu {\bf Zbychowi}. Zależnie od wagi problemu,
rozwiązanie może konsultować z~zarządem.

Członkowie zespołu kontaktują się między sobą telefonicznie i za pomocą sieci
lokalnej, natomiast lider zespołu z zarządem podobnie jak i członkowie zespołu
z przedstawicielami działów kontaktują się przez spotkania osobiste.

\section{Struktura dokumentów}

\section{Kontrola wersji dokumentu}

\end{document}

