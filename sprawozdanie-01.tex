\documentclass[10pt,a4paper]{article}
\usepackage[a4paper]{geometry}

\usepackage{polski}
\usepackage{xltxtra}

\usepackage{fancyvrb}
\usepackage{relsize}
\usepackage[pdfborder={0 0 0}]{hyperref}

%% tweak fonts
\defaultfontfeatures{Mapping=tex-text}
\setromanfont{Charis SIL}
%\setsansfont[Scale=MatchLowercase]{Gill Sans}
%\setmonofont[Scale=MatchLowercase]{Menlo}
\linespread{1.25}

%% define custom commands and environments
\DefineVerbatimEnvironment%
  {SmallVerbatim}%
  {Verbatim}{fontsize=\relsize{-0.5},numbers=left,numbersep=-10pt,frame=lines,tabsize=4}

\begin{document}

%%fakesection{Tytuł}
\title{
  Sprawozdanie nr~1 z~laboratorium\\Podstaw Inżynierii Oprogramowania
}
\author{
  Grzegorz Bartkowiak\\
  Tomasz Cudziło\\
  Mateusz Ochtera\\
  Gustaw Wypych\\
  \\
  \textsc{PW EE Informatyka}\\[10pt]
}
\date{\today}

\maketitle

\section{Sytuacja biznesowa -- Dom aukcyjny \emph{Hammer}}
Zarząd domu aukcyjnego „Hammer” ze względu na postęp techniczny postanowiła
zinformatyzowad swoją firmę. Do tej pory całą dokumentacja finansowa, kadrowa,
spisy przedmiotów do sprzedaży jak i sprzedanych były w postaci papierowej,
przez co praca była trudna, a wydajnośd mała. Zarząd w celu usprawnienia jak i
podniesienia efektywności pracy postanowił przekształcid dokumentacje papierową
na elektroniczne bazy danych. Dodatkowo zarząd postanowił wkroczyd na rynek
internetowy- w tym celu ma powstad internetowy system aukcyjny, z możliwością
brania udziału w aukcjach w czasie rzeczywistym. Chcąc zapewnid maksymalny
sukces swojemu przedsięwzięciu, idąc za radami działu marketingu zarząd
postanowił, że potrzebny jest jak „najświeższy” pomysł młodej grupy ludzi.
Poszukiwania zaczęto od czołowych polskich uczelni technicznych. Po sprawdzeniu
rankingów lat ubiegłych zdecydowano się na Politechnikę Warszawską. Ogłoszono
więc konkurs dla grupy informatyków, która miałaby się zająd tworzeniem takiej
strony internetowej. Chcąc zaoszczędzid na szkoleniu pracowników, jak i w
trosce o klientów postawiono wymogi ze powstałe projekty mają byd przystępne i
łatwe w obsłudze, a operowanie nimi w miarę intuicyjne. Zgodnie z wymogami,
głównymi kryteriami oceniania projektów były: była oryginalnośd, i prostota
mechanizmów serwisu aukcyjnego, czytelnośd jak i	szybkośd	szkieletów baz
danych.. Wymagany był jak najjaśniejszy i przyjazny użytkownikowi interfejs.
Ważny także, był barwny i estetyczny wygląd witryny. Firma ma stworzyd system
baz danych do zarządzania pracownikami, spisami przedmiotów oraz platformą
aukcyjną. Musi także chronid przed oszustwami, oraz mied wysoki poziom
zabezpieczeo przed ewentualnymi zagrożeniami ze strony hakerów. Dodatkowymi
wymaganiami były znajomośd języka angielskiego na wysokim poziomie, znajomośd
języków programowania obiektowego	skryptowego tworzenia grafiki oraz html.

Wygranym zespołem okazali się młodzi studenci Informatyki pierwszego roku na
Wydziale Elektrycznym, których wizjonerski i modernistyczny pomysł wprawił
zarząd w zachwyt.

Informatycy mają za zadanie zaprojektowanie swojego pomysłu oraz dostarczenie
go do działu informatycznego firmy. Dalsza współpraca polegała by na
konsultacjach i wdrożeniu wizji w życie. Na początek firma zażyczyła sobie
projekt baz danych, pracowników i przedmiotów, wykonany przez członków zespołu
odpowiadających za bazy danych. Drugim krokiem współpracy było stworzenie
szablonu platformy aukcyjnej przez grafików. Następnym punktem było stworzenie
wstępnego kodu zarządzającego platformą. Czwartym krokiem jest dodanie kodu
potrzebnego do kontroli próby oszustwa oraz zabezpieczającego przed atakami z
zewnątrz. Na koniec firma życzy sobie przedstawienia testów. Jeśli zespól się
sprawdzi, firma zaoferowała możliwośd zatrudnienia, w celu kontroli i
aktualizacji strony oraz opieki nad bazami danych.

\section{Przygotowanie do projektu}

\section{Zespół projektowy, zadania członków zespołu}

\section{Struktura podziału pracy}
\subsection{Macierz RAM}

\section{Struktura dokumentów}

\section{Kontrola wersji dokumentu}

\end{document}

