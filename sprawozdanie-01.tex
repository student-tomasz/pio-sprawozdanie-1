\documentclass[10pt,a4paper]{article}
\usepackage[a4paper]{geometry}

\usepackage{polski}
\usepackage{xltxtra}

\usepackage{fancyvrb}
\usepackage{relsize}
\usepackage[pdfborder={0 0 0}]{hyperref}

%% tweak fonts
\defaultfontfeatures{Mapping=tex-text}
\setromanfont{Charis SIL}
%\setsansfont[Scale=MatchLowercase]{Gill Sans}
%\setmonofont[Scale=MatchLowercase]{Menlo}
\linespread{1.25}

%% define custom commands and environments
\DefineVerbatimEnvironment%
  {SmallVerbatim}%
  {Verbatim}{fontsize=\relsize{-0.5},numbers=left,numbersep=-10pt,frame=lines,tabsize=4}

\begin{document}

%%fakesection{Tytuł}
\title{
  Sprawozdanie nr~1 z~laboratorium\\Podstaw Inżynierii Oprogramowania
}
\author{
  Grzegorz Bartkowiak\\
  Tomasz Cudziło\\
  Mateusz Ochtera\\
  Gustaw Wypych\\
  \\
  \textsc{PW EE Informatyka}\\[10pt]
}
\date{\today}

\maketitle

\section{Sytuacja biznesowa -- Dom aukcyjny \emph{Hammer}}
Zarząd firmy prowadzącej aukcje internetowe w Wielkiej Brytanii pod domeną
„Hammer.com” postanowił otworzyć oddział swojej firmy w europejskim kraju.
Przeprowadził wywiad na temat sprzedawalności produktów poprzez aukcje
internetowe. Korporacja doszła do wniosku, że najbardziej opłacalnym krajem do
zainwestowania jest Polska. Jako że zwyczaje i przyzwyczajenia Polaków odnośnie
estetyki i wyglądu tego typu stron są różne od brytyjskich, postanowiono nie
korzystać z szablonu obecnej strony, lecz znaleźć inny wizjonerski pomysł na
witrynę . Dział marketingu owej firmy doszedł do wniosku, że potrzebny jest jak
„najświeższy” pomysł młodej grupy ludzi. Poszukiwania zaczęto od czołowych
polskich uczelni technicznych. Po sprawdzeniu rankingów lat ubiegłych
zdecydowano się na Politechnikę Warszawską. Ogłoszono więc konkurs dla grupy
informatyków, która miałaby się zająć tworzeniem takiej strony internetowej.

W badaniach firmy wyszło, że naród polski nie należy do najinteligentniejszych,
dlatego najważniejszym czynnikiem decydującym o tym czy pomysł jest dobry była
oryginalność i prostota mechanizmów sterujących taką stroną. Wymagany był jak
najjaśniejszy i przyjazny użytkownikowi interfejs. Ważny także, był barwny i
estetyczny wygląd witryny.

Firma zajmuje się jedynie udostępnianiem użytkownikowi platformy aukcyjnej, na
której będzie mógł przeprowadzać aukcje, lub w nich uczestniczyć. Musi także
chronić przed oszustwami, oraz mieć wysoki poziom zabezpieczeń przed
ewentualnymi zagrożeniami ze strony hackerów. Poszukiwany był więc zespół,
który miałby współpracować z doświadczoną brytyjską grupą informatyków
zatrudnionych już w firmie. Wymagana była znajomość języka angielskiego na
wysokim poziomie, znajomość jeżyków programowania obiektowego skryptowego
tworzenia grafiki oraz html.

Wygranym zespołem okazali się młodzi studenci Informatyki pierwszego roku na
Wydziale Elektrycznym, których wizjonerski i modernistyczny pomysł wprawił
zarząd w zachwyt.

Informatycy mają za zadanie zaprojektowanie swojego pomysłu oraz dostarczenie
go do działu informatycznego firmy. Dalsza współpraca polegała by na
konsultacjach i wdrożeniu wizji na rynek internetowy. Na początek firma
zażyczyła sobie szablon strony wykonany przez członków zespołu zajmujących się
grafiką. Drugim krokiem współpracy było stworzenie wstępnego interfejsu
aukcyjnego. Trzecim krokiem jest dodanie kodu potrzebnego do kontroli próby
oszustwa oraz zabezpieczającego przed atakami z zewnątrz. Na koniec firma życzy
sobie przedstawienia testów. Jeśli zespól się sprawdzi, firma zaoferowała
możliwość zatrudnienia, w celu kontroli i aktual izacji strony oraz koordynacji
pracy z siedzibą w Anglii.

\section{Przygotowanie do projektu}

\section{Zespół projektowy, zadania członków zespołu}

\section{Struktura podziału pracy}
\subsection{Macierz RAM}

\section{Struktura dokumentów}

\section{Kontrola wersji dokumentu}

\end{document}

