\documentclass[10pt,a4paper]{article}
\usepackage[a4paper]{geometry}

\usepackage{polski}
\usepackage{xltxtra}

\usepackage{fancyvrb}
\usepackage{relsize}
\usepackage[pdfborder={0 0 0}]{hyperref}

%% tweak fonts
\defaultfontfeatures{Mapping=tex-text}
\setromanfont{Charis SIL}
%\setsansfont[Scale=MatchLowercase]{Gill Sans}
%\setmonofont[Scale=MatchLowercase]{Menlo}
\linespread{1.25}

%% define custom commands and environments
\DefineVerbatimEnvironment%
  {SmallVerbatim}%
  {Verbatim}{fontsize=\relsize{-0.5},numbers=left,numbersep=-10pt,frame=lines,tabsize=4}

\begin{document}

%%fakesection{Tytuł}
\title{
  Sprawozdanie nr~1 z~laboratorium\\Podstaw Inżynierii Oprogramowania
}
\author{
  Grzegorz Bartkowiak\\
  Tomasz Cudziło\\
  Mateusz Ochtera\\
  Gustaw Wypych\\
  \\
  \textsc{PW EE Informatyka}\\[10pt]
}
\date{\today}

\maketitle

\section{Sytuacja biznesowa~--- Dom aukcyjny \emph{Hammer}}
Nowy zarząd domu aukcyjnego \emph{Hammer} postanowił zinformatyzować swoje
przedsiębiorstwo. Do tej pory całą dokumentacja finansowa, kadrowa, spisy
przedmiotów do sprzedaży jak i sprzedanych były w postaci papierowej. Cały
proces zarządzania dokumentacją był mało wydajny, a~jego utrzymanie wymagało
sporych nakładów finansowych. Zarząd podjął decyzję zastąpienia starego
systemu, nowym~--- wykonanym na zlecenie~--- całkowicie zinformatyzowanym
systemem, który zarchiwizuje całą, dotychczas zebraną, dokumentację
i~jednocześnie przejmie odpowiedzialność za prowadzenie nowej.

Idąc za ciosem, skoro już wszystko będzie dostępne elektronicznie, kierownik
działu sprzedaży przekonał zarząd do wkroczenia na rynek internetowy. Pod
dotychczasową domeną domu internetowego \emph{hammer.com} ma powstać
internetowy system aukcyjny dający możliwość brania udziału w aukcjach w czasie
rzeczywistym.

Ponieważ system zarządzający dokumentacją nie wymaga niestandardowych
rozwiązań, zarząd zlecił jego wykonanie grupie sprawdzonych informatyków z~ich
siostrzanej firmy handlującej krasnalami ogrodowymi. Natomiast internetowy
system aukcyjny potrzebuje nietuzinkowego podejścia.

Dział marketingu wie, że targetem domu aukcyjnego \emph{Hammer} są młodzi
i~przedsiębiorczy ludzie, do których nie przemawiają stare (pomimo, że
sprawdzone) rozwiązania. Zarząd, wspierając się radami z~marketingu,
postanowił, że kluczem do sukcesu jest jak ``najświeższy'' pomysł. Tylko młoda
grupa ludzi, doświadczona w~używaniu i~tworzeniu oprogramowania,
wykorzystującego nowe technologie na różnych platformach mobilnych
i~desktopowych jest w~stanie stworzyć najlepszy system aukcji internetowych.

Platforma aukcyjna ma udostępniać dotychczasowym i~nowym klientom domu
aukcyjnego \emph{Hammer} przeprowadzanie i~uczestniczenie w~aukcjach. Musi być
bezpieczna i~wygodna w~obsłudze.

Poszukiwania zaczęto od czołowych polskich uczelni technicznych. Po sprawdzeniu
rankingów lat ubiegłych zdecydowano się na Politechnikę Warszawską. Ogłoszono
konkurs mający wyłonić grupę informatyków, która zajmie się zaprojektowaniem,
stworzeniem i~wdrożeniem internetowego systemu aukcyjnego \emph{Hammer.com}.

\noindent Postawiono jasne kryteria oceniania pomysłów konkursowych.
Najważniejszymi były:
\begin{itemize}
  \item prostota i~intuicyjność obsługi systemu od strony pracowników oraz
    klientów,
  \item bezpieczeństwo prowadzonych aukcji.
\end{itemize}
Położono też nacisk na:
\begin{itemize}
  \item oryginalność pomysłów,
  \item wydajność i~małą złożoność wewnętrznych mechanizmów serwisu aukcyjnego,
  \item czytelność graficznego interfejsu użytkownika.
\end{itemize}   

Wygranym zespołem okazali się młodzi studenci informatyki, pierwszego roku na
Wydziale Elektrycznym, których wizjonerski i modernistyczny pomysł wprawił
zarząd w zachwyt.

Zwycięzcy informatycy mają za zadanie zaimplementować i~wdrożyć swój projekt,
we współpracy z~zespołem odpowiedzialnym za wewnętrzny system zarządzający
dokumentacją domu aukcyjnego. Współpraca ma polegać na konsultacjach
i~stopniowym wdrażaniu wizji w~życie:
\begin{enumerate}
  \item przedstawienie logotypu domu aukcyjnego, projektu graficznego
    statycznych stron domu aukcyjnego i~szablonów dla dynamicznych stron
    systemu aukcyjnego,
  \item kod udostępniający dane pracowników, klientów, przedmiotów i~aukcji:
  \begin{enumerate}
    \item zaprojektowanie \texttt{API} do komunikacji zewnętrznego systemu
      aukcyjnego z~wewnętrznym systemem z~danymi pracowników, klientów,
      przedmiotów i~aukcji,
    \item \emph{opcjonalnie} jeśli zaprojektowanie \texttt{API} jest
      nieodpowiednim rozwiązaniem, stworzenie własnego, lokalnego szkieletu baz
      danych.
  \end{enumerate}
  \item kod sterujący aukcjami i~komunikacją z~klientami,
  \item kod zabezpieczający przed działaniami nielegalnymi i/lub~niezgodnymi
    z~regulaminem domu aukcyjnego,
  \item przeprowadzenie testów.
\end{enumerate}

\section{Przygotowanie do projektu}

\section{Zespół projektowy, zadania członków zespołu}

\section{Struktura podziału pracy}
\subsection{Macierz RAM}

\section{Struktura dokumentów}

\section{Kontrola wersji dokumentu}

\end{document}

